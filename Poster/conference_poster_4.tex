%%%%%%%%%%%%%%%%%%%%%%%%%%%%%%%%%%%%%%%%%
% baposter Landscape Poster
% LaTeX Template
% Version 1.0 (11/06/13)
%
% baposter Class Created by:
% Brian Amberg (baposter@brian-amberg.de)
%
% This template has been downloaded from:
% http://www.LaTeXTemplates.com
%
% License:
% CC BY-NC-SA 3.0 (http://creativecommons.org/licenses/by-nc-sa/3.0/)
%
%%%%%%%%%%%%%%%%%%%%%%%%%%%%%%%%%%%%%%%%%

%----------------------------------------------------------------------------------------
%	PACKAGES AND OTHER DOCUMENT CONFIGURATIONS
%----------------------------------------------------------------------------------------

\documentclass[landscape,a0paper,fontscale=0.285]{baposter} % Adjust the font scale/size here

\usepackage{graphicx} % Required for including images
\graphicspath{{figures/}} % Directory in which figures are stored

\usepackage{amsmath} % For typesetting math
\usepackage{amssymb} % Adds new symbols to be used in math mode

\usepackage{booktabs} % Top and bottom rules for tables
\usepackage{enumitem} % Used to reduce itemize/enumerate spacing
\usepackage{palatino} % Use the Palatino font
\usepackage[font=small,labelfont=bf]{caption} % Required for specifying captions to tables and figures

\usepackage{multicol} % Required for multiple columns
\setlength{\columnsep}{1.5em} % Slightly increase the space between columns
\setlength{\columnseprule}{0mm} % No horizontal rule between columns

\usepackage{tikz} % Required for flow chart
\usetikzlibrary{shapes,arrows} % Tikz libraries required for the flow chart in the template

\newcommand{\compresslist}{ % Define a command to reduce spacing within itemize/enumerate environments, this is used right after \begin{itemize} or \begin{enumerate}
\setlength{\itemsep}{1pt}
\setlength{\parskip}{0pt}
\setlength{\parsep}{0pt}
}

\definecolor{lightblue}{rgb}{0.145,0.6666,1} % Defines the color used for content box headers

\begin{document}

\begin{poster}
{
headerborder=closed, % Adds a border around the header of content boxes
colspacing=1em, % Column spacing
bgColorOne=white, % Background color for the gradient on the left side of the poster
bgColorTwo=white, % Background color for the gradient on the right side of the poster
borderColor=lightblue, % Border color
headerColorOne=black, % Background color for the header in the content boxes (left side)
headerColorTwo=lightblue, % Background color for the header in the content boxes (right side)
headerFontColor=white, % Text color for the header text in the content boxes
boxColorOne=white, % Background color of the content boxes
textborder=roundedleft, % Format of the border around content boxes, can be: none, bars, coils, triangles, rectangle, rounded, roundedsmall, roundedright or faded
eyecatcher=true, % Set to false for ignoring the left logo in the title and move the title left
headerheight=0.1\textheight, % Height of the header
headershape=roundedright, % Specify the rounded corner in the content box headers, can be: rectangle, small-rounded, roundedright, roundedleft or rounded
headerfont=\Large\bf\textsc, % Large, bold and sans serif font in the headers of content boxes
%textfont={\setlength{\parindent}{1.5em}}, % Uncomment for paragraph indentation
linewidth=2pt % Width of the border lines around content boxes
}
%----------------------------------------------------------------------------------------
%	TITLE SECTION 
%----------------------------------------------------------------------------------------
%
{\includegraphics[height=7em]{logohtml.png}} % First university/lab logo on the left
{\bf\textsc{Liberia para juegos de mesa SAEM GAMES}\vspace{0.5em}} % Poster title
{\textsc{ Santiago Quintero, Edgar Am\'ezquita y Camilo Ar\'evalo
\newline Pontificia Universidad Javeriana}} % Author names and institution
{\includegraphics[height=7em]{img-logoHeader.png}} % Second university/lab logo on the right

%----------------------------------------------------------------------------------------
%	OBJECTIVES
%----------------------------------------------------------------------------------------

\headerbox{Objetivos}{name=objectives,column=0,row=0}{

\textbf{Saem Games} es un libreria para crear juegos de mesa que tengan un tablero, debe soportar 2 o mas jugadores, Este projecto debe ser implementado en C++ y debe contar con su propia documentacion para el proyecto de objetos y programacion a media escala.\\

\vspace{0.3em} % When there are two boxes, some whitespace may need to be added if the one on the right has more content
}

%----------------------------------------------------------------------------------------
%	INTRODUCTION
%----------------------------------------------------------------------------------------

\headerbox{Analisis}{name=Analisis,column=1,row=0}{

La libreria debe soportar la creacion de diversos juegos de mesa, por tanto la libreria debe poseer objetos que modelen la base del juego, como tablero, jugadores, fichas, cartas y dados. La libreria contendra funciones basicas para un juego, las cuales se pueden extender dependiendo la complejidad del juego. \\
Para muchos juegos un jugador puede poseer varias fichas en un juego, las cuales pueden ser manipuladas agregando, quitando o moviendo las fichas en el respectivo tablero. Ademas que el movimiento de cada ficha debe determinarse por un numero constante relacionado a una regla o ser fijado por el resultado de uno o mas dados contenidos en el juego.\\
No nesesariamente debe haber dados en un juego, pero en el caso de tenerlos, el dado debe generar un numero aleatorio entre un rengo especificado por le usuario.\\



}

%----------------------------------------------------------------------------------------
%	RESULTS 1
%----------------------------------------------------------------------------------------

\headerbox{Conclusion}{name=results,column=2,span=2,row=0,}{

\begin{multicols}{2}

\end{multicols}

%------------------------------------------------

\begin{multicols}{2}


\end{multicols}
}




%----------------------------------------------------------------------------------------
%	CONCLUSION
%----------------------------------------------------------------------------------------

\headerbox{Resultado}{name=conclusion,column=1,span=3,below=Analisis,above=bottom}{

\begin{multicols}{2}
Como resultado podemos ver una libreria con todos sus componentes para poder crear un juego de mesa:
\begin{itemize}\compresslist
\item una clase para las cartas (SCarta), Se definen las funciones para modelar lo basico de una carta, como colocar un nombre a una carta u obtener el nombre de la carta.
\item una clase para las casillas (SCasilla), la clase define funciones para asignar o obtener los atributos de casilla como las cartas o fichas, cuenta con varios constructores para hacer mas facil su construccion.
\item una clase para los dados (SDados), Con esta clase se puede conseguir la funcionalidad de un dado, genera numeros aleatorios entre un numero entero minimo y uno maximo inclusivo.
\item una clase para las fichas (SFicha), La clase define las funciones las funciones para modelar lo basico de una ficha como asignar a un jugador, conseguir o asignar la casilla donde esta la ficha.
\item una clase para el juego en general (SJuego),  Esta Clase es un clase semi abstracta porque toca implementar varias funciones, para que este pueda usarla y se pueda jugar.
\item una clase para los jugadores (sJugador), La clase define las funciones basicas para modelar un jugador, este tiene fichas, un nombre y pertenece a un equipo. Se puede asignar un equipo, fichas, saber a que equipo pertenece y retirar el jugador de ser necesario.
\item una clase para las pila de cartas (SPile\_cartas), Aqui se implementan las funciones basicas de una pila de cartas, como push(), pop(), size() y shuffle().
\item una clase para reglas (SRegla), Es una clase abstracta la cual hay que implementar para poder poner las reglas del juego, que generalmente son reglas de que pasa con las fichas.

\item una clase para el tablero (STablero), Esta clase implementa funciones para conseguir o agregar cartas y fichas.
\end{itemize}

\end{multicols}
}

%----------------------------------------------------------------------------------------
%	MATERIALS AND METHODS
%----------------------------------------------------------------------------------------

\headerbox{Introduccion}{name=introduction,column=0,below=objectives}{

Para este projecto se debia crear una librer\'ia que ayudar\'a al usuario a crear algunos juegos de mesa como escalera, monopoly y otros juegos que maneje un tablero para jugar, aunque esta que la libreria no puede cubrir ciertos tipos de juegos de mesa, tales como risk. Esta libreria nos permitira extender varias de sus clases para ajustarlas a un tipo espec\'ifico de juego y agregar la l\'ogica al juego, ya que la libreria pretende dar una base para crear el juego y en algunos casos no dara abasto para todo el juego.\\

Para probar la funcionalidad de la libreria se realizo un juego de escalera con power ups, en el cual, a diferencia del escalera normal, hay casillas que le dan ciertos poderes a una ficha de un jugador.  


}

\headerbox{Herramientas de Desarrollo}{name=herradesarrollo,column=0,below=introduction,bottomaligned=conclusion}{

En este projecto se hizo uso de:

\begin{enumerate}\compresslist
\item Git Hub como manejador de versiones del projecto.
\item Doxygen para documentacion de codigo.
\item C++ 11 para la implementacion de la libreria.
\item Latex para la realizacion del poster.
\end{enumerate}



}

%----------------------------------------------------------------------------------------
%	RESULTS 2
%----------------------------------------------------------------------------------------

% \headerbox{Results 2}{name=results2,column=1,below=Analisis,bottomaligned=conclusion}{ % This block's bottom aligns with the bottom of the conclusion block


% }

%----------------------------------------------------------------------------------------

%----------------------------------------------------------------------------------------
%	REFERENCES
%----------------------------------------------------------------------------------------

%\headerbox{Referencias}{name=references,column=0,above=bottom}{
%\begin{itemize}\compresslist
%\item template de latex http://www.latextemplates.com/template/baposter-landscape-poster created by Brian Amberg 
%\end{itemize}
%}

\end{poster}

\end{document}